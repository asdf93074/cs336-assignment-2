\documentclass{article}
\usepackage[utf8]{inputenc}
\usepackage[T1]{fontenc}
\usepackage{lmodern}
\usepackage{geometry}
\usepackage{amsmath}
\usepackage{booktabs}
\usepackage{caption}
\usepackage{hyperref}
\usepackage{listings}
\usepackage{graphicx}
\renewcommand{\labelenumi}{(\alph{enumi})}
\lstset{
  basicstyle=\ttfamily\small,
  breaklines=true,
  columns=fullflexible,
  keepspaces=true
}
\geometry{margin=1in}

\title{Benchmarking Notes}
\author{}
\date{}

\begin{document}
\maketitle

\section{benchmarking\_script}

\begin{enumerate}
\item
Written in \texttt{benchmark.py}.

\item
\textbf{Commands:}
\begin{lstlisting}[language=bash]
# small
python benchmark.py --d-model 768 --d-ff 3072 --num-layers 12 --num-heads 12

# medium
python benchmark.py --d-model 1024 --d-ff 4096 --num-layers 24 --num-heads 16

# large
python benchmark.py --d-model 1280 --d-ff 5120 --num-layers 36 --num-heads 20

# xl
python benchmark.py --d-model 1600 --d-ff 6400 --num-layers 48 --num-heads 25

# 2.7B
python benchmark.py --d-model 2560 --d-ff 10240 --num-layers 32 --num-heads 32
\end{lstlisting}


\noindent\textbf{Results:}
\begin{table}[h!]
\centering
\caption{Benchmark results for small and medium models.}
\label{tab:benchmark}
\begin{tabular}{llrrrrr}
\toprule
Model & Pass & Warmup & Mean (s) & Std (s) \\
\midrule
Small  & Forward  & 0 & 0.072297 & 0.121894 \\
Small  & Backward & 0 & 0.106601 & 0.121139 \\
Small  & Forward  & 1 & 0.031458 & 0.000598 \\
Small  & Backward & 1 & 0.067038 & 0.001763 \\
Medium & Forward  & 0 & 0.136340 & 0.117727 \\
Medium & Backward & 0 & 0.241945 & 0.122108 \\
Medium & Forward  & 1 & 0.097735 & 0.001412 \\
Medium & Backward & 1 & 0.202035 & 0.001391 \\
\bottomrule
\end{tabular}
\end{table}

\begin{lstlisting}
Cannot do rest due to memory limitations (8GB).
\end{lstlisting}

\item
Minor increase in measured time with 0 warmup steps. This happens because some optimizations are done based on the first pass, so warming up lets the correct cache/shapes be known in advance for the next passes.
\end{enumerate}

\newpage
\section{nsys\_profile}

\begin{enumerate}
\item
\begin{table}[h!]
\centering
\caption{Benchmark results on forward pass on Nsys vs Python standard library.}
\label{tab:benchmark}
\begin{tabular}{llrrrrr}
\toprule
Model & Context Length & Nsys Mean (ms) & Python Mean (ms) \\
\midrule
Small  & 128 & 40.780  & 40.479 \\
Small  & 256 & 41.320  & 43.825 \\
Small  & 512 & 40.996  & 70.388 \\
\bottomrule
\end{tabular}
\end{table}

Total time is roughly 40\,ms for all context sizes (did small model only due to memory constraints) but our measured time in Python keeps increasing due to device sync overhead.

\item
\texttt{ampere\_sgemm\_128x64\_nn} takes up the most time in both forward and backward passes. It is called 52 times in the forward pass.

\item
\textbf{Forward:}
\begin{lstlisting}
Time    Total Time    Instances    Avg         Med         Min         Max         StdDev    Name
5.6%    2.167 ms      94           23.053 \textmus   22.688 \textmus   21.920 \textmus   30.304 \textmus   1.260 \textmus  void at::native::elementwise_kernel<(int)128, (int)2, void at::native::gpu_kernel_impl_nocast<...>>
\end{lstlisting}

\noindent\textbf{Backward:}
\begin{lstlisting}
Time     Total Time    Instances    Avg         Med         Min         Max         StdDev    Name
13.7%    7.872 ms      11           715.670 \textmus  597.540 \textmus  592.869 \textmus  1.914 ms    397.501 \textmus void cutlass::Kernel2<cutlass_80_simt_sgemm_128x64_8x5_nt_align1>(T1::Params)
10.5%    6.012 ms      68           88.410 \textmus   36.176 \textmus   1.152 \textmus    274.627 \textmus  85.005 \textmus  void at::native::vectorized_elementwise_kernel<(int)4, ...>
\end{lstlisting}

\item
Optimizer takes up a huge chunk of time but overall, kernel contribution remains the same.

\item
Matrix multiplication takes approximately 762\,$\mu$s while computing softmax takes approximately 800\,$\mu$s. The matrix multiplication has much more FLOPs than softmax.
\end{enumerate}

\newpage
\section{mixed\_precision\_accumulation}

\begin{lstlisting}
ans. 
tensor(10.0001)
tensor(9.9531, dtype=torch.float16)
tensor(10.0021)
tensor(10.0021)
\end{lstlisting}

Accumulating in FP32 lets us retain a more accurate result when adding floats of lower precision, regardless of whether we upscale the lower precision float or not.

\newpage
\section{benchmarking\_mixed\_precision}

\begin{enumerate}
\item
\begin{itemize}
  \item Model parameters: FP32
  \item Output of first feedforward layer: FP16
  \item Output of layer norm: FP16
  \item Predicted logits: FP16
  \item Loss: FP32
  \item Gradients: FP16
\end{itemize}

\item
The mean and variance calculations in layernorm are sensitive to mixed precision.
The subtraction in mean, squaring in variance and sqrt in normalization are all sensitive.
Using BF16 is okay as it has the same range as FP32 so we could treat layernorm with mixed precision then.


\item
With bigger batch sizes, model sizes and context length, mixed precision is significantly faster.
\end{enumerate}

\newpage
\section{benchmarking\_mixed\_precision}
Using only small + 128 context length due to memory limitations.

\begin{enumerate}
\item
\begin{figure}[h]
    \centering
    \includegraphics[scale=0.3]{../traces/images/memory_viz_full_small_128_forward_only.png}
    \caption{Forward only.}
\end{figure}
\begin{figure}[h]
    \centering
    \includegraphics[scale=0.3]{../traces/images/memory_viz_full_small_128_full_step.png}
    \caption{Full step.}
\end{figure}

The peaks for forward pass have very sharp ends which makes them identifiable. Memory rises and falls for each pass. We also see a big chunk being cleared when we are clearing gradients in our full step.

\item
Peak memory usage of Forward Pass: ~3.4GB
Peak memory usage of Full Step: ~4.7GB

\item
Peak memory usage (mixed precision) of Forward Pass: ~3GB
Peak memory usage (mixed precision) of Full Step: ~4.5GB

There is a small improvement in memory usage but not a major one.

\item
Considering for small sized model: (using 16 batch size) 16 x 128 x 768 x 2 bytes = 0.75 MiB 

\item
Yes by calculating the memory size of a matrix at a particular layer, we actually can find it in the graph and also figure out where we are in the taining layers. Just looking at the biggest allocations also helps figure this out e.g. in my case its the logits shape (16 x 128 x 10000) which indicates the end of training when its allocation ends.
We can also look at some of the smaller repetitive patterns to see our transformer layers in action.
\end{enumerate}

\newpage
\section{pytorch\_attention}
\begin{enumerate}
\item
\begin{table}[h!]
\centering
\label{tab:benchmark}
\begin{tabular}{llrrrrr}
\toprule
d\_model & Sequence Length & Forward Mean (ms) & Backward Mean (ms) & Mem. Usage (MB) \\
\midrule
16 & 256 & 0.3331 & 2.2741 & 41.8000 \\
16 & 1024 & 0.4399 & 1.7989 & 239.3200 \\
16 & 4096 & 0.5644 & 16.4094 & 3139.7600 \\
32 & 256 & 0.3786 & 1.0753 & 339.9800 \\
32 & 1024 & 0.4001 & 1.8448 & 245.2800 \\
32 & 4096 & 0.5684 & 16.5382 & 3145.7600 \\
64 & 256 & 0.3932 & 1.1707 & 369.9800 \\
64 & 1024 & 0.4139 & 1.8072 & 239.3600 \\
64 & 4096 & 0.5717 & 16.9213 & 3175.7600 \\
128 & 256 & 0.6964 & 2.0495 & 459.9200 \\
128 & 1024 & 0.4726 & 1.9223 & 273.0400 \\
128 & 4096 & 0.5614 & 19.2572 & 3701.5200 \\
\bottomrule
\end{tabular}
\end{table}
OOM on sequence lengths 8192 and 16384 for all d\_models.
For any QK where sequence length is 8192:
  batch size x sequence length x sequence length = 8 x 8192 x 8192 x 4 bytes = 2048 MB

We need 2x that for the backward pass.

Memory reserved before backward pass is being dominated by sequence length. We can see that regardless of d\_model, a sequence length of 4096 results in memory usage upwards of 3GB. Reducing the quadratic memory nature of QK in attention would help in reducing memory significantly.

\end{enumerate}

\newpage
\section{torch\_compile}
\begin{enumerate}
\item
\begin{table}[h!]
\centering
\label{tab:benchmark}
\begin{tabular}{llrrrrr}
\toprule
d\_model & Sequence Length & Forward Mean (ms) & Backward Mean (ms) & Forward Mean C. (ms) & Backward Mean C. (ms) &  \\
\midrule
16 & 256 & 0.3331 & 2.2741 &  0.2479 & 3.7499 & 41.8000 \\
16 & 1024 & 0.4399 & 1.7989 & 0.3416 & 1.0678 & 175.3200 \\
16 & 4096 & 0.5644 & 16.4094 & 0.5069 & 6.6920 & 2115.7600 \\
32 & 256 & 0.3786 & 1.0753 &  0.3094 & 1.8125 & 339.9800 \\
32 & 1024 & 0.4001 & 1.8448 & 0.3598 & 1.1687 & 181.2800 \\
32 & 4096 & 0.5684 & 16.5382 & 0.5129 & 8.6241 & 2621.5200 \\
64 & 256 & 0.3932 & 1.1707 &  0.2981 & 1.8196 & 389.9800 \\
64 & 1024 & 0.4139 & 1.8072 & 0.3559 & 1.3961 & 206.7200 \\
64 & 4096 & 0.5717 & 16.9213 & 0.5458 & 9.8528 & 2653.5200 \\
128 & 256 & 0.6964 & 2.0495 & 0.2965 & 0.7804 & 467.9200 \\
128 & 1024 & 0.4726 & 1.9223  & 0.2800 & 1.4377 & 228.5200 \\
128 & 4096 & 0.5614 & 19.2572 & 0.5270 & 16.1470 & 2729.0400 \\
\bottomrule
\end{tabular}
\end{table}

\item
Memory issues so cannot try it out.

\end{enumerate}

\end{document}
